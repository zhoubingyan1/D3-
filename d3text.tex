1.选择集(写成链式语法) 
	d3.select("").
	d3.selectAll().
	d3.text();
2.选择元素
	d3.select();是选择所有指定元素的第一个
	d3.selectAll();是指定元素的全部
3.绑定数据
	datum():绑定一个数据到选择集上
	data():绑定一个数组到选择集上,数组的各项值分别与选择集的各元素绑定
4.选择、插入、删除元素
	选择: select()、selectAll()
	插入: append()末尾添加元素、insert()元素前添加
	删除: remove()删除元素
5.画布
	SVG :
		1.SVG 绘制的是矢量图,因此对图像进行放大不会失真。
		2.基于 XML,可以为每个元素添加 JavaScript 事件处理器。
		3.每个图形均视为对象,更改对象的属性,图形也会改变。
		4.不适合游戏应用。
	Canvas:
		1.绘制的是位图,图像放大后会失真。
		2.不支持事件处理器。
		3.能够以 .png 或 .jpg 格式保存图像
		4.适合游戏应用

6.比例尺的使用
	d3.scale.linear()	返回一个线性比例尺
	domain()	设定比例尺的定义
	range()	设定比例尺的值域
		例:比例尺的定义域 domain 为:[0.9, 3.3]
			比例尺的值域 range 为:[0, 300]
			因此,当输入 0.9 时,返回 0;当输入 3.3 时,返回 300。当输入 2.3 时呢?返回 175,这是按照线性函数的规则计算的。
	d3.max()	求数组最大值
	d3.min()	求数组最小值

7.坐标轴
	d3.svg.axis() D3中坐标轴的组件,能够在svg中生成坐标轴的元素
	scale() 	指定比例尺
	orient() 	指定刻度的朝向,bottom表示在坐标轴的下方显示
	ticks()		指定刻度的数量

	

